\documentclass{article}

\usepackage{amsmath, amsfonts, amssymb, amsthm}
\usepackage{cancel}
\usepackage{tcolorbox}

\begin{document}

\title{A random formula with logs of primes}
\author{dnzc}
\date{}

\maketitle

\tableofcontents
\newpage



\section{Introduction}

We will derive the following fact:

$\forall a,b,p,q \in \mathbb{N}$ with $p,q$ distinct primes,

\[ab = \sum_{x=1}^a \text{min}(b, \lfloor x \log_q p\rfloor) + \sum_{x=1}^b \text{min}(a, \lfloor x \log_p q\rfloor)\]
This formula was derived by thinking about the following problem:\\\\
Let $n=2^{31}3^{19}$. How many positive divisors of $n^2$ are less than $n$ but do not divide $n$?\\\\
We will solve a more general version of the above problem in two different ways, which will lead us to the main result arrived at in the Summary section.


\section{Solving the Problem}
The version of the problem that we will think about is as follows:
For some $a,b,p,q \in \mathbb{N}$ with $p,q$ distinct primes, let $n = p^{a}q^{b}$. How many positive divisors of $n^2$ are less than $n$ but do not divide $n$?


\subsection{Computational Way}

The factors of $n^2$ are in the form $p^r q^s, \,\,r,s \in \mathbb{Z}$ with $0 \leq r \leq 2a, 0 \leq q \leq 2b$\\
If this factor does not divide $n$ then we must have $r>a$ or $s>b$.\\
We cannot have both $r>a$ and $s>b$ since that would imply $p^r q^s > n$ and we are counting the factors that are less than n.
Thus these cases are disjoint and we can count them separately.

\subsubsection{Case 1}
We will count the case when $r>a$.\\
Then more specifically, $a < r \leq 2a$ and $0 \leq s < b$. And:
\begin{equation*}
\begin{split}
    p^r q^s < p^a q^b\\
    \implies p^{r-a} < q^{b-s}
\end{split}
\end{equation*}
Let $x = r-a, y=b-s$\\
Then with some inequality manipulations we get $1 \leq x \leq a$ and $1 \leq y \leq b$
\begin{equation*}
\begin{split}
    p^x < q^y\\
    \implies x \log_q p < y \text{ (log rules)}\\
    \implies \lfloor x \log_q p \rfloor + 1 \leq y \leq b \text{ (} \log_q p \text{ is irrational)}
\end{split}
\end{equation*}
So for any given x, the number of possibilities for y contributed to the count is
\[\text{max}(0, b - \lfloor x\log_q p\rfloor)\]
We use the "max" function here to contribute 0 if $\lfloor x \log_q p\rfloor \geq b$ rather than contributing a negative number (since there are 0 possibilities for y in that case)\\
So, the total number of possibilities counted by Case 1 is the sum of this over all possible values of x:
\[ = \sum_{x=1}^a \text{max}(0, b - \lfloor x \log_q p\rfloor)\]

\subsubsection{Case 2}
The case when $s>b$ is symmetric to Case 1, with p,q swapped and a,b swapped but the argument is entirely the same.\\
Thus similarly to Case 1, the count contributed by this case is
\[ \sum_{x=1}^b \text{max}(0, a - \lfloor x \log_p q\rfloor)\]

\subsubsection{Total}
So the overall answer to the problem is the sum of Case 1 and Case 2, which is
\begin{equation*}
\begin{split}
    \sum_{x=1}^a \text{max}(0, b - \lfloor x \log_q p\rfloor) + \sum_{x=1}^b \text{max}(0, a - \lfloor x \log_p q\rfloor)\\
    = \sum_{x=1}^a \left[ b - \text{min}(b, \lfloor x \log_q p\rfloor) \right] + \sum_{x=1}^b \left[ a - \text{min}(a, \lfloor x \log_p q\rfloor) \right] \\
    = 2ab - \left[ \sum_{x=1}^a \text{min}(b, \lfloor x \log_q p\rfloor) + \sum_{x=1}^b \text{min}(a, \lfloor x \log_p q\rfloor) \right]
\end{split}
\end{equation*}
\begin{tcolorbox}
\begin{equation}
\begin{split}
    \therefore \text{answer} = 2ab - \left[ \sum_{x=1}^a \text{min}(b, \lfloor x \log_q p\rfloor) + \sum_{x=1}^b \text{min}(a, \lfloor x \log_p q\rfloor) \right]
\end{split}
\end{equation}
\end{tcolorbox}





\subsection{Combinatorial Way}

We will show that the overall answer is simply $ab$.\\\\
$n^2 = p^{2a}q^{2b}$ so there are $(2a+1)(2b+1)$ factors of $n^2$.\\\\
We can consider positive factor pairs of $n^2$ - one number of the pair will $<n$ and the other will be $>n$, except for the case when the factor pair is $(n, n)$. In other words, if $a$ is a positive factor of $n^2$ and $a<n$ then $\frac{n^2}{a} > n$ so exactly one of the factor pair $(a, \frac{n^2}{a})$ will be less than $n$.\\\\
Thus the number of factors of $n^2$ that are less than $n$ is
\[ \# \text{ factors of }n^2 \text{ less than }n = \frac{(2a+1)(2b+1) - 1}{2} = 2ab + a + b\]
where we exclude the $(n, n)$ case and then halve.\\\\

Now,
\begin{equation*}
\begin{split}
    \# \text{ factors of } n^2 \text{ less than } n \text{ AND divide }n\\
    (\# \text{ factors of } n) - 1 \text{  (since we exclude n as a factor of n)}\\
    = (a+1)(b+1) - 1 = ab + a + b
\end{split}
\end{equation*}

Finally,
\begin{equation*}
\begin{split}
    \text{answer = } \# \text{ factors of } n^2 \text{ less than } n \text{ AND NOT divide }n\\
    = \# \text{ factors of }n^2 \text{ less than }n - 
    \# \text{ factors of } n^2 \text{ less than } n \text{ AND divide }n\\
    = (2ab + a + b) - (ab + a + b)\\
    = ab
\end{split}
\end{equation*}
\begin{tcolorbox}
\begin{equation}
\begin{split}
    \therefore \text{answer} = ab
\end{split}
\end{equation}
\end{tcolorbox}



\section{Summary} 
Equating (1) and (2):
\[ab = 2ab - \left[ \sum_{x=1}^a \text{min}(b, \lfloor x \log_q p\rfloor) + \sum_{x=1}^b \text{min}(a, \lfloor x \log_p q\rfloor) \right]\]
Rearrange to get
\begin{tcolorbox}
\begin{equation*}
\begin{split}
    ab = \sum_{x=1}^a \text{min}(b, \lfloor x \log_q p\rfloor) + \sum_{x=1}^b \text{min}(a, \lfloor x \log_p q\rfloor)\\
    \forall a,b,p,q \in \mathbb{N} \text{ with } p,q \text{ distinct primes}
\end{split}
\end{equation*}
\end{tcolorbox}
We have shown our result.



\section{Conjecture} 
If we replace $\log_q p$ with the positive irrational constant c, then we have a conjecture:
\begin{tcolorbox}
\begin{equation*}
\begin{split}
    \forall a,b \in \mathbb{N}, c \in \mathbb{R} \setminus \mathbb{Q}, c>0\\
    ab = \sum_{x=1}^a \text{min}(b, \big\lfloor cx\big\rfloor) + \sum_{x=1}^b \text{min}(a, \big\lfloor \frac{x}{c} \big\rfloor)\\
\end{split}
\end{equation*}
\end{tcolorbox}

Then we note that this can be easily proven true by counting the lattice points inside a rectangle cut through by a line (similar to the pretty picture lemma). More specifically:\\\\
Consider the line y=cx and the rectangle with vertices (0,0), (a+1, 0), (b+1,0) (a+1, b+1).\\\\
We will count the number of lattice points strictly inside the rectangle in two different ways. (Note c is irrational so no lattice points lie on y=cx)\\\\
One way is directly width*height = a*b.\\
The other way is counting the lattice points that are inside the rectangle and underneath the line y=cx by summing for each value of x, and then doing the same thing for y (i.e. the rest of the points, to the left of y=cx).\\\\
Doing this, we arrive at the conjecture, QED.

\end{document}